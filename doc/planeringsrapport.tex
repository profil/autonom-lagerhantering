\documentclass[a4paper,11pt]{article}

\usepackage[swedish]{babel}
\usepackage[T1]{fontenc}
\usepackage[utf8]{inputenc}
\usepackage[lighttt]{lmodern}
\usepackage{parskip}
%\usepackage{amsmath}
%\usepackage{amssymb}
%\usepackage{amsthm}
%\usepackage{listings}
%\usepackage{graphicx}
\usepackage{biblatex}
\usepackage{csquotes}
\usepackage{pgfgantt}
\usepackage[a4paper]{geometry}


\addbibresource{references.bib}

\author{Daniel Eriksson \and Daniel Pettersson \and Gunnar Bolmvall 
\and Jonathan Hasselström \and Pedro Josefsson \and Victor Andersson}

\title{Autonom Lagerhantering \\ Planeringsrapport}

\begin{document}
\maketitle


\section{Bakgrund}
Den industriella revolutionen sträcker sig från tidigt 1700-tal till idag
där människan har, i samklang med naturen och tekniska möjligheter,
utvecklat och förbättrat tekniker i syfte att skapa robustare system med
högre produktivitet. Allt från ''Spinning Jenny'' till James Watts ångmaskin
har förenklat människans vardag och tillfälligt stillat den eviga strävan
efter ständiga förbättringar som genomsyrar människans långa historia
\cite{organisationochorganisering}.

Vad gäller effektiva materialflöden inom produktion tog Henry Ford det
historiskt viktiga steget när han införde löpande-band-principen i sin
bilfabrik, en förändring som kom att ligga till grund för de flesta mass-
och serieproduktionssystem idag. Ford utmanade gängse normer genom
banbrytande idéer för att vägleda monteringsobjekten genom fabriken förbi
montörer och operatörer, så kallad linelayout. Senare vidareutvecklades
drivning av linan till att bli helt automatiserad och därigenom mer
resurseffektiv \cite{organisationochorganisering}.

Ytterligare ett steg i utvecklingen är att tillåta en automatiserad
materialförsörjning i allt från lagermiljöer till massproduktionslinor.
Istället för en manuell materialhantering där lagerarbetare och truckförare
rör sig genom lagermiljöer för att försörja monterings- eller
packningsstationer med rätt material och produkter så automatiseras
processen.

Enligt Toyotas teorier om LEAN production är rörelser i ett system en av de
\emph{7+1 slöserierna} och bör undvikas i största möjliga mån. Genom att
automatisera materialförsörjningen minimeras de mänskliga rörelserna vilket
medför att en större del av det mänskliga arbetet åtgår till värdeskapande
aktiviteter \cite{organisationochorganisering}. Tidigare års
kandidatarbeten som berör ämnet grundades på en AGV\footnote{Automated
Guided Vehicle} som autonomt åker mellan lagerplatser och annan godtyckligt
definierad position. Dessa arbeten ligger till grund för ett fortsatt
arbete med projektet där fokus ligger på att använda den kunskap som
förvärvats och utveckla den för att nå längre och uppnå bättre resultat.
 

\section{Syfte}
Avsikten med projektet är att genom vidareutveckling av föregående års
kandidatarbeten utarbeta och implementera en lösning för autonom
materialhantering med hjälp av AGV:er. Med hjälp av ett överordnat system
ska AGV:er koordineras till att transportera förutbestämd lagerinredning
till bestämda platser för att till exempel en montör ska slippa hämta detta
själv. Systemet ska vara möjligt att implementera i ett lagersystem med
flera AGV:er och varje AGV utrustas med ett kollisionssystem för att
undvika hinder. Målet är att öka effektiviteten och på ett effektivt och
tillförlitligt sätt minimera kostnader samt reducera personskador och höja
ergonomin i anknytning till materialhantering i lagermiljöer. 


\section{Problem/Uppgift}
Huvudproblemet grundas i att en AGV autonomt ska kunna navigera mellan
kända positioner i en känd miljö, där den ska ha möjlighet att utifrån ett
överordnat system ta order och genomföra leveranser. AGV:n ska även undvika
kollision med fasta och rörliga objekt, samt vara utrustad med en anordning
som möjliggör materialtransport. För att lösa dessa problem, delas systemet
upp i följande delsystem:

\begin{enumerate}
  \item Lagerinredningen skall vara utformad så att
  \begin{enumerate}
    \item Den på ett säkert sätt skall kunna transporteras med AGV:n
  \end{enumerate}

  \item Plockstationen skall
  \begin{enumerate}
    \item Ge order till det överordnade systemet om vad som skall hämtas
		via ett lämpligt gränssnitt
    \item Godkänna när rätt order blivit plockad så att AGV:n kan återlämna lagerhyllan
  \end{enumerate}

  \item Navigationssystem skall
  \begin{enumerate}
    \item Möjliggöra ett system i lagerlokalen som AGV:n kan navigera efter
	\item Vara kommersiellt gångbart
  \end{enumerate}

  \item AGV:n skall
  \begin{enumerate}
    \item Navigera till rätt plats
    \item Utrustas med en anordning som möjliggör transport av lagerhyllan
    \item Kontrollera att rätt lagerhylla hämtas
    \item Undvika kollision med rörliga och stationära hinder
    \item Placera lagerhyllan på angiven plats, inom rimlig marginal
  \end{enumerate}

  \item Överordnade system skall
  \begin{enumerate}
    \item Hålla koll på vilken order som skall hämtas
    \item Ruttplanera AGV:ns väg till och från målet
    \item Undvika att kollision med andra AGV:er uppstår
	\item Tillhandahålla ett gränssnitt som möjligör orderhantering och
		uppdaterar användaren med status av orderhämtning.
    \item Hålla reda på lagerhyllornas position
  \end{enumerate}
\end{enumerate}



\section{Avgränsningar}

För att kunna hålla projektet inom given tidsram måste lämpliga
avgränsningar i projektuppgiften göras. Till en början avgränsas arbetet
till att behandla en AGV men detta skall dock inte begränsa det överordnade
systemet, som bör konstrueras för att hantera fler AGV:er.

På liknande sätt kommer gruppen att utforma lagermiljön så att den
innehåller endast en operatör även om det överordnade systemet skall kunna
hantera flera operatörer. Detta för att spara tid samt minimera och
fokusera uppgiften på att först få den huvudsakliga funktionen, hämta och
lämna lagerinredning, att fungera korrekt. Här avgränsar gruppen sig till
att fortsätta med föregående års tanke med att lagerinredningen skall ha en
förutbestämd struktur och design.

En annan begränsning är självfallet att projektet måste utföras med
tillgänglig hårdvara och budget. Gruppen kommer även endast arbeta med en,
för det överordnade systemet, känd miljö för att kunna optimera
ruttplaneringen. Miljön avgränsas till att verka i enbart ett plan och
ordnas så att det ej existerar hinder på höjder som inte kan detekteras

Två stora områden som gruppen väljer att avgränsa sig emot är felplacering
av lagerinredning samt lagersaldohållning. Felplacering av lagerinredning
innebär alltså att lagerinredningen förflyttas manuellt vilket gör att
systemet tappar kontrollen på var de befinner sig i lokalen. Gruppen
förutsätter här att all personal är väl införstådd med systembegränsningen.
Gällande lagersaldohållningen är det troligt att det redan finns en extern
saldohållningsfunktion hos företag som kan tänkas använda den framtagna
AGV:n vilket gör utvecklandet av en sådan funktion överflödig och
tidsödande.

Gruppen överväger i dagsläget att avgränsa sig till navigering via QR-koder
efter en snabbare genomgång av föregående års rapporter. Detta då gruppen
funnit brister i användning av lasernavigering och en låg potential vid
eventuell utvidgning av systemet vid granskning av det arbete som tidigare
gjorts med AGV:erna. Detta ska dock utredas mer ingående innan beslut
fattas \cite{kivasystems}.

\section{Genomförande}
Projektet var från den uppdragsgivande institutionen avsett att bygga
vidare på tidigare års kandidatarbeten, dock påvisade dessa bristfälliga
resultat och ytterligare frågetecken har framkommit vid en första
granskning av arbetena \cite{laser,qr}. På grund av detta behöver gruppen
ta ett beslut kring huruvida utveckling av ett nytt koncept bör ske
alternativt fortsättning av något av ovan nämnda arbeten. Detta beslut är
avgörande för hur projektet kommer att arta sig och hur långt gruppen
kommer hinna. Nedan följer mer ingående hur gruppen avser att gå tillväga.

\subsection{Förstudie}
I och med att projektet syftar till att vidareutveckla befintliga
kandidatarbeten avser gruppen att först kritiskt granska dessa. Utöver
detta bör gruppen även försöka införskaffa sig så mycket information som
möjligt från institutionen på Chalmers och företag i branschen för att
skapa sig en bild av vad som är realiserbar samt kommersiellt gångbart.

Gruppen studerar kandidatarbetena först för att bredda sin kunskap inom
ämnet och med förhoppning om att finna delar i arbetet man vill gå vidare
med. I de fall gruppen avser att använda sig av befintligt material
förlitar gruppen sig på att lämpliga avvägningar och studier genomförts,
här är klart fördelaktigt om det inom tidigare projekt testats och
realiserats. Ytterligare en faktor inför eventuella beslut kring
vidareutveckling bör vara tillgången av det material gruppen avser att
utveckla.

\subsection{Konkretisering}
I samband med förstudien bör gruppen på ett strukturerat sätt analysera
problemet och finna de nyckelfunktioner som behöver lösas. Framtagning av
kravspecifikation bör gruppen även göra i detta skede. För att få fram en
kommersiellt gångbar lösning är det av vikt att gruppen tillsammans med
tilltänkta brukare specificerar kraven. Samtliga krav skall här kopplas med
en typ av verifiering och i största mån, ett kvantitativt mål.

\subsection{Konceptlösning}
I projektets första fas avser gruppen, förutom att konkretisera problemet,
att även sammanställa en generell konceptlösning som löser problemen i
största möjliga mån. Koncepten baseras på uppkommet stoff ur
litteraturstudien, från samtal med fackmän och ingenjörsmässiga avväganden
från gruppen. För att kontrollera konceptens möjlighet till realisering bör
en utvärdering göras via samtal med handledare och yrkesverksamma.


\subsection{Realisering}
Vid ett godkänt koncept avser sedan gruppen att realisera detta med en
prototyp. Realiseringsarbetet kommer ske genom diverse ombyggnationer av
den befintliga AGV:n. Då detta är ett konstruktionsarbete avser gruppen att
flytande använda sig av de stöd som finns att tillgå; manualer, studie- och
facklitteratur samt stöd från handledare och yrkesverksamma inom området.
Detta för att kunna lösa problem i takt med att de uppkommer. Arbetsgången
under realiseringen antas i flera anseenden vara iterativ. Gruppen avser
att kontinuerligt testa och utvärdera konstruktionsarbetet och därmed
kontinuerligt förändra och förbättra tills godkänt resultat uppnås.

\subsection{Utvärdering}
För att vid slutet av projektet kunna utvärdera hur väl uppsatta mål
avklarats utformas vid ett tidigt stadie ett helhetstest som syftar att
kontrollera samtliga funktioner hos systemet. Testet bör i största mån
efterlikna den miljö som AGV:n är tänkt att verka i och uppgifterna likaså.
I de fall som kraven på funktionen i kravspecifikationen är av kvantitativa
mått kan ett avskärmat test genomföras. Detta funktionstest avser att
avgöra om projektet är lyckat eller inte. I det stora hela anses skall
helhetsfunktionen vara avgörande vid utvärdering.

\subsection{Redovisning}
Arbetet som gruppen utför kommer att redovisas i dels två olika muntliga
presentationer men även en vetenskaplig rapport. Utöver detta skall också
arbetet, som tidigare nämnts, mynna ut i en fungerande AGV som skall kunna
demonstreras.



\section{Tidsplan}
För att få en överskådlig bild av projektets olika delmoment och
inlämningar uppförs en tabell (se tabell~\ref{tab:planeringstabell})
innehållandes de moment som det finns fasta datum för samt ett Ganttschema
(se figur~\ref{fig:gantt}) med en planering för arbetsgången. Planeringen
av arbetsgången är gjord utifrån grova uppskattningar med hänsyn till hur
lång tid de olika delarna kommer att ta och genom att uppföra Ganttschemat
fås en tydlig bild av att arbetet med olika delar sker parallellt.

\begin{table}[h]
  \begin{center}
    \begin{tabular}{l | l}
      Aktivitet                           & Datum      \\ \hline
      Planeringsrapport                   & 2015-02-10 \\
      Handledningstillfälle 1             & LP3 LV5    \\
      Mittmöte                            & 2015-03-16 \\
      Handledningstillfälle 2             & LP4 LV3    \\
      Handledningstillfälle 3             & LP4 LV6    \\
      Sammanställd slutrapport            & 2015-05-19 \\
      Skriftlig individuell opposition    & 2015-05-19 \\
      Presentation och muntlig opposition & 2015-05-28 \\
    \end{tabular}
  \end{center}
  \caption{Obligatoriska planerade moment.}
  \label{tab:planeringstabell}
\end{table}

\newpage

\newgeometry{left=1.8cm}
\begin{figure}[h]
  \begin{ganttchart}[
    canvas/.style={draw=black!15},
    vgrid={*1{draw=black!15}},
    title/.style={draw=black!15},
    title label font=\bfseries\footnotesize,
    today=7,
    today label={Deadline planering},
    title height=1,
    y unit title=0.5cm]{4}{22}
    \gantttitle[
      title label node/.append style={left=-6pt}]
      {Vecka:\quad4}{1}
    \gantttitlelist{5,...,22}{1} \\
    \gantttitle{LP3}{9}
    \gantttitle{LP4}{10} \\
    \gantttitle[
      title label node/.append style={left=-6pt}]
      {Läsvecka:\quad1}{1}
    \gantttitlelist{2,...,8}{1}
    \gantttitle{T}{1}
    \gantttitlelist{1,2}{1}
    \gantttitle{P}{1}
    \gantttitle{T}{1}
    \gantttitlelist{3,...,8}{1} \\

    \ganttbar{Planeringsrapport}{5}{7} \\
    \ganttbar{Förkovra oss i ämnet}{8}{10} \\
    \ganttbar{Genomgång av AGV}{8}{10} \\
    \ganttbar{Navigationsystem}{10}{19} \\
    \ganttbar{Kollisionsystem}{12}{19} \\
    \ganttbar{Design av överordnat system}{11}{19} \\
    \ganttbar{Framtagandet av lyftanordning}{15}{19} \\
    \ganttbar{Individuell opposition}{20}{21} \\
    \ganttbar{Slutrapport}{8}{21} \\
    \ganttbar{Presentation}{20}{22}
  \end{ganttchart}
  \caption{Ganttschema över tidsplaneringen.}
  \label{fig:gantt}
\end{figure}
\restoregeometry

\newpage
\printbibliography

\end{document}

