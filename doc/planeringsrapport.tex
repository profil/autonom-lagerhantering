\documentclass[a4paper,11pt]{article}

\usepackage[swedish]{babel}
\usepackage[T1]{fontenc}
\usepackage[utf8]{inputenc}
\usepackage[lighttt]{lmodern}
\usepackage{parskip}
%\usepackage{amsmath}
%\usepackage{amssymb}
%\usepackage{amsthm}
%\usepackage{listings}
%\usepackage{graphicx}
\usepackage{biblatex}
\usepackage{csquotes}
\usepackage{pgfgantt}
\usepackage[a4paper]{geometry}
\usepackage{hyperref}


\addbibresource{references.bib}

\author{Daniel Eriksson \and Daniel Pettersson \and Gunnar Bolmvall 
\and Jonathan Hasselström \and Pedro Josefsson \and Victor Andersson}

\title{Autonom Lagerhantering \\ Planeringsrapport}

\begin{document}
\maketitle


\section{Bakgrund}
Att hushålla med resurser är en viktig del av företagande oavsett om det
rör sig om pengar, ytor eller miljön. Inom tillverkningsindustrin, som har
ett starkt fäste i Sverige, hålls det ofta stora lager vilket binder
kapital och tar upp stor platsyta. Detta görs dels för att företag ska
kunna hålla höga servicenivåer men även för att den ständigt föränderliga
marknaden är svår att förutspå. Stora svenska företag arbetar ofta aktivt
med att minska dessa kostnader genom att utarbeta effektiva
logistiklösningar, exempelvis via ''Supply chain management''\cite{log}.
Problemet med detta är att man oftast behandlar in- och utflöden från lager
och fabriker. Hänsyn tas således inte till de faktiska materialflödena inom
lagret.

Oftast sker materialhanteringen i lagret manuellt med hjälpmedel som
truckar, pallyftare och vagnar. Hos de företag som kommit längre i sin
automatisering används ibland paternosterverk\footnote{Paternosterverk är
en automatiserad lagerhylla.}, men dessa är ofta inte flexibla gällande nya
lagerlösningar och är begränsade till små artiklar\cite{log}.
 
Konsultbolaget Boston Consulting Group menar att det finns möjligheter att
minska kostnaderna inom industrin med 16\% genom en ökning av
automatiseringen under de kommande 10 åren\cite{bcg}. Med tanke på
materialhanteringens stora manuella arbete finns det kostnadsminskningar
att göra även här.
 
En uppmärksammad automationslösning för lagerhantering har utvecklats av
företaget Kiva Systems. Det baseras på ett integrerat system där
operatörer, AGV:er\footnote{Automatic Guided Vehicle} och lagerhyllor
samspelar. Kortfattat utgörs systemet av flexibla lagerplatser där ett
överordnat system övervakar och håller koll på inventarier. Under systemet
verkar individuella AGV:er med uppgift att transportera lagerhyllorna
mellan operatörerna och lagerplatsen. Detta system möjliggör en flexiblare
användning av lagerytorna då utrymmet lätt kan omorganiseras\cite{kivasystems}.
Den största vinsten finns i kostnadsbesparingar och
effektivitetshöjningar. Utöver detta förbättras även operatörernas
arbetsbörda samtidigt som risken för personskador minskar\cite{truckar}. 

Som grund för fortsatt arbete ligger tidigare års kandidatarbete där syftet
var att automatiska AGV:er skulle transportera lagerinredning åt ett
överordnat system\cite{qr,laser}. Tidsbrist har medfört att projekten inte
nått önskat resultat och det är i ljuset av detta årets kandidatarbete tar
vid för att utveckla och uppnå ett mer tillfredställande resultat. 


\section{Syfte}
Avsikten med projektet är att genom vidareutveckling av föregående års
kandidatarbeten utarbeta och implementera en lösning för autonom
materialhantering med hjälp av AGV:er. Med hjälp av ett överordnat system
ska AGV:er koordineras till att transportera förutbestämd lagerinredning
till bestämda platser för att till exempel en montör ska slippa hämta detta
själv. Systemet ska vara möjligt att implementera i ett lagersystem med
flera AGV:er och varje AGV utrustas med ett kollisionssystem för att
undvika hinder. Målet är att öka effektiviteten och på ett effektivt och
tillförlitligt sätt minimera kostnader samt reducera personskador och höja
ergonomin i anknytning till materialhantering i lagermiljöer. 


\section{Problem/Uppgift}
Avsikten med projektet är att genom vidareutveckling av föregående års
kandidatarbeten utarbeta och implementera en lösning för autonom
materialhantering med hjälp av AGV:er. Med hjälp av ett överordnat system
ska AGV:er koordineras till att transportera förutbestämd lagerinredning
till bestämda platser för att till exempel en montör ska slippa hämta detta
själv. Systemet ska vara möjligt att implementera i ett lagersystem med
flera AGV:er och varje AGV utrustas med ett kollisionssystem för att
undvika hinder. Målet är att öka effektiviteten och på ett effektivt och
tillförlitligt sätt minimera kostnader samt reducera personskador och höja
ergonomin i anknytning till materialhantering i lagermiljöer. 

\begin{enumerate}
  \item Lagerinredningen skall vara utformad så att
  \begin{enumerate}
    \item Den skall kunna transporteras av AGV:n utan att varor går sönder
  \end{enumerate}

  \item Plockstationen skall
  \begin{enumerate}
    \item Ge order till det överordnade systemet om vad som skall hämtas
		via ett lämpligt gränssnitt
    \item Godkänna när rätt order blivit plockad så att AGV:n kan återlämna lagerhyllan
  \end{enumerate}

  \item Navigationssystem skall
  \begin{enumerate}
    \item Möjliggöra ett system i lagerlokalen som AGV:n kan navigera efter
	\item Vara grund till ett kommersiellt gångbart system
  \end{enumerate}

  \item AGV:n skall
  \begin{enumerate}
    \item Navigera till rätt plats
    \item Utrustas med en anordning som möjliggör transport av lagerhyllan
    \item Kontrollera att rätt lagerhylla hämtas
    \item Undvika kollision med rörliga och stationära hinder
    \item Placera lagerhyllan på angiven plats, inom rimlig marginal
  \end{enumerate}

  \item Överordnade system skall
  \begin{enumerate}
    \item Hålla koll på vilken order som skall hämtas
    \item Ruttplanera AGV:ns väg till och från målet
    \item Undvika att kollision med andra AGV:er uppstår
	\item Tillhandahålla ett gränssnitt som möjligör orderhantering och
		uppdaterar användaren med status av orderhämtning.
    \item Hålla reda på lagerhyllornas position
  \end{enumerate}
\end{enumerate}

Det är svårt att uppskatta vilka delar som kommer kräva störst arbetsinsats
och vilka som kommer kräva mest tid men föregående kandidatarbeten ger en
fingervisning om var tidskrävande problem kan uppstå. Fingervisningen tyder
på att navigationssystemet och det överordnade systemet är tunga, och för
helheten viktiga, delar av arbetet vilket troligen leder till att mycket
tid kommer läggas på dessa bitar. Eftersom det dock finns tillgång till
tidigare arbeten bör utvecklingen av ovanstående system underlättas något,
till skillnad från utvecklandet av ett lämpligt gränssnitt för användning
av AGV:n som inte skapats tidigare. 

Att tidsuppskatta skapandet av ett gränssnitt är komplicerat eftersom det
inte finns någon kunskap att tillgå kring dess omfattning vilket innebär en
risk att det kan visa sig vara problematiskt att genomföra. Troligtvis
kommer det att vara svårt att finna vetenskaplig fakta som fullt stödjer
besluten och i många fall kommer troligtvis stor vikt ligga på samtal med
handledare och fackmän. Att ständigt behöva vända sig till andra personer
för faktasökning är något som kan bli både tidskrävande och problematiskt.

Utöver detta är det en utmaning att finna ett koncept som passar. Det finns
i dagsläget många olika sorters lager och en automationslösning kan vara
svår att anpassa till flera olika system. Tekniska problem kommer att
uppstå oavsett lösning, men om lösningen är för avgränsad kommer det vara
svårt att implementera på riktiga lagersystem. Detta är särskilt
problematiskt då det kommer innefatta en avvägning om det framarbetade
konceptet är tillräckligt brett för att kunna ligga till grund för ett
neutralt, och för ett godtyckligt lagersystem, fungerade koncept.

Design av anordning för transport av lagerhylla beräknas inte vara lika
tidskrävande och svår i jämförelse med navigationssystemet och det
överordnade systemet, främst med tanke på den kompetens som finns att
tillgå men också tillgången till de lösningsförslag från tidigare års
kandidatarbeten.

Alternativa navigationssystem granskas eftersom det uppdagats brister i
användningen av lasernavigation till det givna problemet, baserat på en
undersökning av arbeten som tidigare gjorts med AGV:erna. Exempelvis är
lasersystemet känsligt för de många störningar som uppkommer vid
arbetshöjden, bland annat att människor och hyllben blockerar
reflektionspunkterna och därmed dess lokaliseringsförmåga vilket omöjliggör
en högeffektiv transport\cite{laser}. Dessa problem skapar incitament för
en utvärdering av alternativa koncept vad gäller navigationen av AGV:n.


\section{Avgränsningar}
För att kunna hålla projektet inom given tidsram måste lämpliga
avgränsningar i projektuppgiften göras. Till en början avgränsas arbetet
till att enbart behandla en AGV. Detta skall dock inte begränsa det
överordnade systemet, som bör konstrueras för att hantera fler AGV:er. 

På liknande sätt kommer lagermiljön utformas så att den endast innehåller
en operatör. Detta för att spara tid samt minimera och fokusera uppgiften
på att först få den huvudsakliga funktionen, hämta och lämna
lagerinredning, att fungera korrekt. Här beslutas det att fortsätta med
föregående års tanke med att lagerinredningen skall ha en förutbestämd
struktur och design.

För att optimera ruttplaneringen antas den tillgängliga arbetsmiljön vara
känd för det överordnade systemet. Miljön avgränsas också till att verka i
enbart ett plan och ordnas så att det ej existerar hinder på höjder som
inte kan detekteras eftersom det inte tillhör det primära problemet.

Två stora områden som projektet avgränsas emot är felplacering av
lagerinredning samt lagersaldohållning. Felplacering av lagerinredning
innebär alltså att lagerinredningen förflyttas manuellt vilket gör att
systemet tappar kontrollen på var de befinner sig i lokalen. Det förutsätts
här att all personal är väl införstådd med systembegränsningen. Gällande
lagersaldohållningen är det troligt att det redan finns en extern
saldohållningsfunktion hos företag som kan tänkas använda den framtagna
AGV:n vilket gör utvecklandet av en sådan funktion överflödig och
tidsödande.



\section{Genomförande}
Projektet var från den uppdragsgivande institutionen avsett att bygga
vidare på tidigare års kandidatarbeten, dock påvisade dessa bristfälliga
resultat och ytterligare frågetecken har framkommit vid en första
granskning av arbetena\cite{laser,qr}. På grund av detta behövs ett beslut
fattas kring huruvida utveckling av ett nytt koncept bör ske alternativt
fortsättning av något av ovan nämnda arbeten. Detta beslut är avgörande för
hur projektet kommer att arta sig och hur många delmål som kommer hinna
uppfyllas. Nedan följer mer ingående hur projektet kommer gå tillväga.


\subsection{Förstudie}
I och med att projektet syftar till att vidareutveckla befintliga
kandidatarbeten avses att först kritiskt granska dessa. Utöver detta bör så
mycket information som möjligt införskaffas från institutionen på Chalmers
och företag i branschen för att skapa sig en bild av vad som är
realiserbart.

De tidigare kandidatarbetena studeras för att bredda kunskapen inom ämnet
för att finna delar värda att ta till vara och arbeta vidare med. I de fall
befintligt material används antages att lämpliga avvägningar och studier
genomförts, här är det klart fördelaktigt om det inom tidigare projekt
testats och realiserats. Ytterligare en faktor inför eventuella beslut
kring vidareutveckling bör vara tillgången av det material som avses
utvecklas.


\subsection{Konkretisering}
I samband med förstudien bör problemet analyseras på ett strukturerat sätt
för att på så vis finna de nyckelfunktioner som behöver lösas. Framtagning
av kravspecifikation bör även göras i detta skede. Samtliga krav skall här
kopplas med en typ av verifiering och i största mån, ett kvantitativt mål.



\subsection{Konceptlösning}
I projektets första fas avses, förutom att konkretisera problemet, att även
sammanställa en generell konceptlösning som löser problemen i största
möjliga mån. Koncepten baseras på litteraturstudien, samtal med fackmän och
ingenjörsmässiga avväganden. För att kontrollera konceptens möjlighet till
realisering bör en utvärdering göras via samtal med handledare och
yrkesverksamma.


\subsection{Realisering}
Vid ett godkänt koncept avses en prototyp tas fram för att realisera
projektet. Realiseringsarbetet kommer ske genom diverse ombyggnationer av
den befintliga AGV:n. Då detta är ett konstruktionsarbete avses att
flytande använda sig av de stöd som finns att tillgå; manualer, studie- och
facklitteratur samt stöd från handledare och yrkesverksamma inom området.
Detta för att kunna lösa problem i takt med att de uppkommer. Arbetsgången
under realiseringen antas i flera avseenden vara iterativ. Projektet avser
att kontinuerligt testa och utvärdera konstruktionsarbetet och därmed
kontinuerligt förändra och förbättra tills godkänt resultat uppnås.


\subsection{Utvärdering}
För att vid slutet av projektet kunna utvärdera hur väl uppsatta mål
avklarats utformas vid ett tidigt stadie ett helhetstest som syftar att
kontrollera samtliga funktioner hos systemet. Testet bör i största mån
efterlikna den miljö som AGV:n är tänkt att verka i och uppgifterna likaså.
I de fall som kraven på funktionen i kravspecifikationen är av kvantitativa
mått kan ett avskärmat test genomföras. Detta funktionstest avser att
avgöra om projektet är lyckat eller inte. I det stora hela anses
helhetsfunktionen vara avgörande vid utvärdering.


\subsection{Redovisning}
Arbetet som utförs kommer att redovisas i dels två olika muntliga
presentationer men även en vetenskaplig rapport. Utöver detta skall också
arbetet, som tidigare nämnts, mynna ut i en fungerande AGV som skall kunna
demonstreras.


\section{Tidsplan}
För att få en överskådlig bild av projektets olika delmoment och
inlämningar har en tabell (se tabell~\ref{tab:planeringstabell}) uppförts,
innehållande de moment som det finns fasta datum för, samt även ett
Ganttschema (se figur~\ref{fig:gantt}) med en planering för arbetsgången.
Planeringen av arbetsgången är gjord utifrån grova uppskattningar med
hänsyn till hur svåra och därmed hur tidskrävande de olika delarna är. 

En fördjupning i ämnet och analyser av tidigare års rapporter krävs för att
kunna utveckla systemet inom den satta tidsramen. Det krävs även en
genomgång av materialet så som hårdvara och mjukvara för att skaffa sig en
bild av hur lösningar kan genomföras rent praktiskt.

För att arbeta kontinuerligt med projektets olika delar som skall
integreras (navigationssystem, kollisionssystem etcetera) kommer
utvecklandet av dessa delsystem ske parallellt vilket syns tydligt i
Ganttschemat. Skälet till att jobba med de olika delarna samtidigt är för
att effektivisera utvecklandet och ha en löpande feedback om delarna för
att underlätta integrering så att inte en funktion förutsätts i ett annat
delsystem som sedan ej går att realisera.

Navigationssystemet ses som den mest tidskrävande delen då tidigare
rapporter visar på att komplexiteten hos problemet samt på de tekniska
instrumenten är väldigt hög. Utvecklingen av ett gränssnitt är en mer
tidsmässigt svåruppskattad del eftersom det inte finns tidigare arbeten
inom ämnet att tillgå samt att komplexiteten är okänd. Ytterligare en stor
del i arbetet är utvecklingen av det överordnade systemet, då de viktigaste
funktionerna (exkluderat navigationssystemet) kommer behandlas där. 

Ett autonomt lagersystem är inte komplett utan ett fungerade
kollisionssystem samt en förmåga att kunna transportera lagerinredning.
Enligt tidigare arbeten så visar det sig att kollisionssystemet är
krångligt och tar mycket tid\cite{laser,qr}. Framtagande av en anordning
som möjliggör transporten av lagerhyllan förmodas även det vara en
omfattande del men ta mindre tid i anspråk i jämförelse med bland annat
navigationssystemet.

Slutrapporten kommer vara ett kontinuerligt arbete under hela tidsspannet
för att underlätta rapportering och dokumentering av vad som gjorts samt
vilka resultat som har tagits fram.



\begin{table}[h]
  \begin{center}
    \begin{tabular}{l | l}
      Aktivitet                           & Datum      \\ \hline
      Planeringsrapport                   & 2015-02-10 \\
      Handledningstillfälle 1             & LP3 LV5    \\
      Mittmöte                            & 2015-03-16 \\
      Handledningstillfälle 2             & LP4 LV3    \\
      Handledningstillfälle 3             & LP4 LV6    \\
      Sammanställd slutrapport            & 2015-05-19 \\
      Skriftlig individuell opposition    & 2015-05-19 \\
      Presentation och muntlig opposition & 2015-05-28 \\
    \end{tabular}
  \end{center}
  \caption{Obligatoriska planerade moment.}
  \label{tab:planeringstabell}
\end{table}

\newpage

\newgeometry{left=1.8cm}
\begin{figure}[h]
  \begin{ganttchart}[
    canvas/.style={draw=black!15},
    vgrid={*1{draw=black!15}},
    title/.style={draw=black!15},
    title label font=\bfseries\footnotesize,
    today=7,
    today label={Deadline planering},
    title height=1,
    y unit title=0.5cm]{4}{22}
    \gantttitle[
      title label node/.append style={left=-6pt}]
      {Vecka:\quad4}{1}
    \gantttitlelist{5,...,22}{1} \\
    \gantttitle{LP3}{9}
    \gantttitle{LP4}{10} \\
    \gantttitle[
      title label node/.append style={left=-6pt}]
      {Läsvecka:\quad1}{1}
    \gantttitlelist{2,...,8}{1}
    \gantttitle{T}{1}
    \gantttitlelist{1,2}{1}
    \gantttitle{P}{1}
    \gantttitle{T}{1}
    \gantttitlelist{3,...,8}{1} \\

    \ganttbar{Planeringsrapport}{5}{7} \\
    \ganttbar{Förkovra oss i ämnet}{8}{10} \\
    \ganttbar{Genomgång av AGV}{8}{10} \\
    \ganttbar{Navigationsystem}{10}{19} \\
    \ganttbar{Kollisionsystem}{12}{19} \\
    \ganttbar{Design av överordnat system}{11}{19} \\
    \ganttbar{Utveckling av gränssnitt}{15}{19} \\
    \ganttbar{Anordning för transport av lagerhylla}{15}{19} \\
    \ganttbar{Individuell opposition}{20}{21} \\
    \ganttbar{Slutrapport}{8}{21} \\
    \ganttbar{Presentation}{20}{22}
  \end{ganttchart}
  \caption{Ganttschema över tidsplaneringen.}
  \label{fig:gantt}
\end{figure}
\restoregeometry

\newpage
\printbibliography

\end{document}

